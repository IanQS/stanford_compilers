%%%%%%%%%%%%%%%%%%%%%%%%%%%%%%%%%%%%%%%%%
% Structured General Purpose Assignment
% LaTeX Template
%
% This template has been downloaded from:
% http://www.latextemplates.com
%
% Original author:
% Ted Pavlic (http://www.tedpavlic.com)
%
% Note:
% The \lipsum[#] commands throughout this template generate dummy text
% to fill the template out. These commands should all be removed when
% writing assignment content.
%
%%%%%%%%%%%%%%%%%%%%%%%%%%%%%%%%%%%%%%%%%

%----------------------------------------------------------------------------------------
%	PACKAGES AND OTHER DOCUMENT CONFIGURATIONS
%----------------------------------------------------------------------------------------

\documentclass{article}

\usepackage{fancyhdr} % Required for custom headers
\usepackage{lastpage} % Required to determine the last page for the footer
\usepackage{extramarks} % Required for headers and footers
\usepackage{graphicx} % Required to insert images
\usepackage{lipsum} % Used for inserting dummy 'Lorem ipsum' text into the
\usepackage{ textcomp }

\usepackage{hyperref}
\hypersetup{
    colorlinks=true,
    linkcolor=blue,
    filecolor=magenta,
    urlcolor=cyan,
}
% template

\usepackage{amsmath}

% Margins
\topmargin=-0.45in
\evensidemargin=0in
\oddsidemargin=0in
\textwidth=6.5in
\textheight=9.0in
\headsep=0.25in

\linespread{1.1} % Line spacing

% Set up the header and footer
\pagestyle{fancy}
\lhead{\hmwkAuthorName} % Top left header
\chead{\hmwkClass\ (\hmwkTitle)} % Top center header
\rhead{\firstxmark} % Top right header
\lfoot{\lastxmark} % Bottom left footer
\cfoot{} % Bottom center footer
\rfoot{Page\ \thepage\ of\ \pageref{LastPage}} % Bottom right footer
\renewcommand\headrulewidth{0.4pt} % Size of the header rule
\renewcommand\footrulewidth{0.4pt} % Size of the footer rule

\setlength\parindent{0pt} % Removes all indentation from paragraphs

%----------------------------------------------------------------------------------------
%	DOCUMENT STRUCTURE COMMANDS
%	Skip this unless you know what you're doing
%----------------------------------------------------------------------------------------

% Header and footer for when a page split occurs within a problem environment
\newcommand{\enterProblemHeader}[1]{
\nobreak\extramarks{#1}{#1 continued on next page\ldots}\nobreak
\nobreak\extramarks{#1 (continued)}{#1 continued on next page\ldots}\nobreak
}

% Header and footer for when a page split occurs between problem environments
\newcommand{\exitProblemHeader}[1]{
\nobreak\extramarks{#1 (continued)}{#1 continued on next page\ldots}\nobreak
\nobreak\extramarks{#1}{}\nobreak
}

\setcounter{secnumdepth}{0} % Removes default section numbers
\newcounter{homeworkProblemCounter} % Creates a counter to keep track of the number of problems

\newcommand{\homeworkProblemName}{}
\newenvironment{homeworkProblem}[1][Topic \arabic{homeworkProblemCounter}]{ % Makes a new environment called homeworkProblem which takes 1 argument (custom name) but the default is "Problem #"
\stepcounter{homeworkProblemCounter} % Increase counter for number of problems
\renewcommand{\homeworkProblemName}{#1} % Assign \homeworkProblemName the name of the problem
\section{\homeworkProblemName} % Make a section in the document with the custom problem count
\enterProblemHeader{\homeworkProblemName} % Header and footer within the environment
}{
\exitProblemHeader{\homeworkProblemName} % Header and footer after the environment
}

\newcommand{\problemAnswer}[1]{ % Defines the problem answer command with the content as the only argument
\noindent\framebox[\columnwidth][c]{\begin{minipage}{0.98\columnwidth}#1\end{minipage}} % Makes the box around the problem answer and puts the content inside
}

\newcommand{\homeworkSectionName}{}
\newenvironment{homeworkSection}[1]{ % New environment for sections within homework problems, takes 1 argument - the name of the section
\renewcommand{\homeworkSectionName}{#1} % Assign \homeworkSectionName to the name of the section from the environment argument
\subsection{\homeworkSectionName} % Make a subsection with the custom name of the subsection
\enterProblemHeader{\homeworkProblemName\ [\homeworkSectionName]} % Header and footer within the environment
}{
\enterProblemHeader{\homeworkProblemName} % Header and footer after the environment
}

%----------------------------------------------------------------------------------------
%	NAME AND CLASS SECTION
%----------------------------------------------------------------------------------------

\newcommand{\hmwkClass}{Compilers} % Course/class
\newcommand{\hmwkTitle}{Week 3 - Parsing and Top-Down parsing} % Assignment title
\newcommand{\hmwkClassTime}{-} % Class/lecture time
\newcommand{\hmwkAuthorName}{Ian Quah (itq)} % Your name

%----------------------------------------------------------------------------------------
%	TITLE PAGE
%----------------------------------------------------------------------------------------

\title{
\vspace{2in}
\textmd{\textbf{\hmwkClass:\ \hmwkTitle}}\\
\vspace{3in}
}

\author{\textbf{\hmwkAuthorName}}

%----------------------------------------------------------------------------------------

\begin{document}

\maketitle
\newpage
%----------------------------------------------------------------------------------------
%	PROBLEM 1
%----------------------------------------------------------------------------------------

% To have just one problem per page, simply put a \clearpage after each problem

\begin{homeworkProblem}{\textbf{Parsing - Introduction}}

  \problemAnswer{

    \textbf{Regular Languages}
    \begin{enumerate}
    \item Weakest formal languages - most languages aren't regular

      \{($^i)^i$ $|$ i $\geq$ 0 \}, the set of all balanced parens, which can't be
      represented as a regular language

    \item What \textbf{CAN} a regular language express?
      \begin{enumerate}
      \item count mod k

        can't calculate to some arbitrarily high value, like in balanced parens problem
      \end{enumerate}

    \item
      \begin{tabular}{c| c| c}
        Phase & Input & Output\\
        \hline
        Lexer & String of Chars & String of Tokens\\
        Parser & String of tokens (output of lexer) & Parse Tree\\
      \end{tabular}
    \end{enumerate}
  }
\end{homeworkProblem}

\begin{homeworkProblem}{\textbf{Context Free Grammars}}

  \begin{enumerate}
  \item \textbf{The Problem}

    \begin{enumerate}
    \item Not all strings of tokens are programs

      Thus, need a language for describing valid strings of tokens and method to
      distinguish invalid and valid strings of tokens

      Note:
      CFG $\supset$ RegExp

    \item Programming Languages have a recursive structure, and CFGs are a
      natural notation for describing these structures.
    \end{enumerate}

  \item \textbf{What IS a CFG?}

    \begin{enumerate}
    \item A set of terminals, T
    \item A set of non-terminals, N
    \item A start symbol, S (S $\in$ N)
    \item A set of productions: (A symbol, followed by an arrow then a list of symbols)

      X $\rightarrow$ Y$_1$, ... Y$_n$

      X $\in$ N and $Y_i \in N \cup T \cup \{\epsilon\}$
    \end{enumerate}

  \item \textbf{CFG example: Balanced Parens Problem}

    \textbf{Productions / Rules}

    \problemAnswer{
      S $\rightarrow$ (S) \hfill S, our start state then becomes another state with two
      parens around it

      S $\rightarrow$ $\epsilon$
    }

    \textbf{Our States}

    N = \{S\}

    T = \{(, )\}

    S = start symbol \hfill in general the first production will specify the
    start symbol for that CFG

  \item \textbf{Productions as rules:}

    \begin{enumerate}
    \item Begin with a String that only has start symbol, S
    \item Replace any non-terminal X in start by RHS of some production
    \item Repeat (2) until there are no non-terminals
    \end{enumerate}

    If we consider a single step as $\alpha$, then a program can be thought of as

    $\alpha_0$ $\rightarrow$ $\alpha_1$ $\rightarrow$ ... $\rightarrow$ $\alpha_n$ \hfill equiv $\alpha_0$ $\rightarrow^*$ $\alpha_n$


  \item \textbf{CFG formal definition}

    Let G be a CFG with start symbol S, then L(G) of G is:

    \[\{ a_1,...,a_n | \forall i , a_i \in T, S\rightarrow^* a_1,...,a_n \}\]


  \end{enumerate}
\end{homeworkProblem}

\begin{homeworkProblem}{\textbf{Derivations Part 1}}
  \begin{enumerate}
  \item \textbf{Derivation:} a sequence of productions
  \item \textbf{Representations}

    - A derivation can be drawn as a tree instead of a linear path

    - A derivation can be drawn as a tree

    \begin{enumerate}
    \item Start symbol is the root
    \item For a production, add the children
    \end{enumerate}


    \textbf{Example: }

    Grammar:  E $\rightarrow$ E + E $|$ E * E $|$ (E) $|$ id

    \includegraphics[width=13cm]{derivation_tree}

    LHS: Derivation intermediates (Left-most derivation)

    RHS: Parse tree of input string

  \item \textbf{Parse Trees}
    \begin{enumerate}
    \item Terminals at the leaves
    \item Non-terminals at interiors
    \item in-order traversal of tree is original input
    \item Parse tree shows association of operations, input string does not (in
      tree we see multiplication is more tight than addition)
    \item Left-most and right-most derivation produces same parse tree, but the
      main difference is the intermediate step
    \end{enumerate}

  \end{enumerate}
\end{homeworkProblem}

\begin{homeworkProblem}{\textbf{Ambiguity in CFGs}}

  \begin{enumerate}
  \item \textbf{Ambiguity}: if there is more than one parse tree for some string
    (more than 1 left-most or right-most derivation)

    This is bad, because it means that the decision is left to the compiler

  \item Determining if it's ambiguous

    Construct a string, then work backwards and see if there is more than 1 way
    to get the tree

  \item \textbf{Removing ambiguity}

    id * id + id

    E $\rightarrow$ E' + E | E'

    E' $\rightarrow$ id * E' | id | (E) * E' | (E)

    rewrite to enforce precedence of * over +, by separating multiplication and
    + s.t * is handled by E'
  \end{enumerate}
\end{homeworkProblem}


\end{document}
